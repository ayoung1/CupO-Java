\documentclass[12pt]{article}
\usepackage{fancyhdr}
\setlength{\headheight}{15.2pt}
\pagestyle{fancy}
\fancyhead[l]{SRS}
\fancyhead[c]{Cup of Java}
\fancyhead[r]{EWU CSCD 350}
%\fancyfoot[r]{Eastern Washington University}
\begin{document}
\noindent
\thispagestyle{empty}
\LARGE \\
Trivia Maze\\
Software Requirement Specifications\\
by Cup of Java\\

\pagebreak

\noindent
\normalsize
\textbf{\Large 1.0 Introduction}

This document will lay out the requirements, goals, and constraints for Cup of Java's Trivia Maze project for CSCD 350.\\

\textbf{\Large1.1 Project Requirements}

Outlined by Tom. Time frame, additional objectives\\

Create a maze traversable by the user from entrance to exit. The maze will be composed of rooms containing one or more doors. To move into a new room, the user must answer a trivia question. Incorrectly answered questions will lock that particular door. The game is won on completion of the maze, and lost when no path to the end exists.\\

Questions will be numbered, of multiple types (short answer, T/F, multiple choice) and stored in a SQLite database.\\

There will be a visual representation of the maze for the user to interact with (possibly multiple).\\

Variations on a theme are welcome (items to help the user, finite use powers, etc).\\

Min size of maze is for rooms (no max specified). Entrance/exit may be random or fixed at opposite ends.\\

User must be able to save/load (serialize game state)\\

Provide admin tools for database editing\\

Scrub user input\\

Extra points for audio/video in question\\

Unit Tests required\\

OO principles, patterns where appropriate, and
Interfaces where productive.\\

\textbf{\Large 1.2 Additional Goals}

Host the database online, allowing for a question entry portal and admin access.\\

Scrape, format, and add questions from trivia sites such as j-archive.com (Jeopardy).\\

Cheats for debugging.\\

Random (solvable) maze generation\\

Maze difficulty (paths to exit, categories/question difficulty)\\

\textbf{\Large 1.3 Scope}

Overlaps with above...\\


\noindent
\textbf{\Large 2.0 Usage Scenario}\\

\noindent
\textbf{\Large 3.0 Data Model}\\

\noindent
\textbf{\Large 4.0 Functional Model}\\

\noindent
\textbf{\Large 5.0 Behavioral Model}\\

\noindent
\textbf{\Large 6.0 Validation Criteria}\\

\noindent
\textbf{\Large 7.0 Appendices}\\
Template: http://www.rspa.com/docs/Reqmspec.html

\end{document}